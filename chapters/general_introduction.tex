Cranberries (\textit{Vaccinium macrocarpon}) are a unique and valuable fruit native to North America, widely recognized for their distinctive tart flavor, vivid red color, and nutritional benefits. Rich in antioxidants, vitamins, and dietary fiber \cite{blumberg_impact_2016}, cranberries have long been celebrated for their potential health benefits such as supporting cardiovascular function and providing anti-inflammatory properties \cite {leahy_cranberry-promising_2001}. Due to these benefits, a surge in consumer demand for cranberry-based products, ranging from fresh berries and juices to supplements and extracts. To help fill this demand, cranberry producers have shown an increasing interest in the incorporation of new technology in multiple areas of production \cite{jackson_case_2021}, from crop maintenance and pest control to harvest and processing \cite{johnson_ml_cranberry_2008}. 

While mechanization and precision farming techniques have become ubiquitous, particularly in grain and row-crop farming \cite{hameed_technology_2025}. Large-scale use is easier to implement and costs are spread onto thousands of acres and integrated into fewer and larger machines used across all those acres. However, the use of such technology has been slower in crops used for food production, such as vegetables and berry production, because crop and field sizes are generally smaller and economies of scale are more difficult to realize.

Fortunately, as technology has evolved in recent decades, consumer use has pushed robotics and mechatronics into smaller and smaller form factors and, importantly, used this widespread use to reduce production and implementation costs. Power storage in the form of new generations of batteries paired with economical forms of generation such as solar has given robotics the ability to perform more power-intensive jobs, or the ability to do those jobs over extended durations.  On the other side of the power equation, more compact and powerful motors and actuators have been able to more efficiently use this energy provided to accomplish a wide variety of tasks from crop harvesting \cite{das_designing_2024} to weeding and seeding operations \cite{vahdanjoo_operational_2023}

Alongside consumer use and demand for smaller, more powerful mechatronics, hobbyist development of electronic control for systems ranging from drones to 3D printing has made the level of entry into precision computerized control of mechatronics systems a smaller hurdle when developing small-scale systems. No longer are large multi-disciplined teams of software and electronics engineers needed, but systems can be built up from preassembled parts of open-sourced software with wider support from internet-based communities.

Although machine learning has been used in other fruiting crops for both fruit and flower detection, it is generally done at one time during flowering \cite{lee_detection_2021} to estimate the fruit count or is used after fruit development to help predict crop ripeness \cite{villacres_detection_2020}. This study aims to improve on those works by increasing the sampling rate to multiple times during the flowering to allow for tracking both flower count and duration as well as including a connection between flowering and fruit set by collecting data both during and after flowering to record fruit set and early fruit development. In order to produce consistent data in a controlled way, automation is used to both decrease sampling intervals and reduce variables such as image timing, image resolution, orientation, and ground sampling distance. This will be accomplished by using a purpose-built automated ground vehicle (AGV) that is able to traverse the cranberry bog autonomously.  This AGV will be able to sample and track images and store the data.


