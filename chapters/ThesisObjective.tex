This thesis aims to provide an example of the use of automation, crop imaging, and machine learning as a system to better understand and predict plant flowering and fruit set. Flowering is a crucial phase in the cranberry growth cycle, as it directly influences berry formation, yield, and overall crop quality. Understanding the complex dynamics of cranberry flowering, including factors such as floral initiation, development, and pollination, is essential to optimize cultivation practices and improve productivity.  This process is influenced by a range of factors, including the environment and genetics. Plant nutrients, precipitation levels, and temperature also represent external factors that affect the timing and success of flowering.

Specifically, we will build and document an autonomous imaging platform and survey workflow that will train and compare modern detectors (YOLOv11/YOLOv12/RF-DETR/RF3) on a labeled dataset ($\geq$1,400 images; $\geq$50k instances) for flower/early-fruit detection. After training, we will select a deployment-feasible model meeting mAP@0.50 $\geq$ 68\% and single-image latency $\leq$ 75~ms at $640 \times 640$, and finally test the association between total flower count (aggregated over a standardized bloom window) and total berry count using OLS and cross-validated predictive error, with pre-specified subgroup analyses by cultivar (Stevens vs. Experimental hybrids).

As a result of this study, better methods for analyzing and tracking the timing and extent of cranberry flowering will offer producers the ability to make more informed and timely decisions about crop management. In addition to crop production, an improved yield estimate will allow for more efficient downstream supply chain planning and marketing. Furthermore, this project will aim to demonstrate the potential use and future expansion of autonomous ground vehicles to collect and process image data, as well as to incorporate additional sensors to further improve crop management and yield prediction.
